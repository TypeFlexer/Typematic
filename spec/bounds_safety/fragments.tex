

\chapter{Random Fragments}

Here are some fragments of ideas that need to be more fully developed.

\section{Discussion of optimizing bounds check and overflow}

We had the following discussion in Section~{section:bounds-checking-at-indirections}
that distracted from the
main point of the section. It's also confusing the overflow check for x
+ c is omitted. The point of the example below is that we can make code
efficient. That deserves a fuller discussion and should be buttressed by
empirical data.

Consider as an example, z = *(x+5); where x : bounds(x, x + c). The
compiler will produce code of the form

\begin{lstlisting}
assert(x != null);
t1 = x + 5;
check x + 5 for overflow
assert(t1 != null && x <= t1 && t1 < x + c);
z = *t1;
\end{lstlisting}

This simplifies to:

\begin{lstlisting}
assert(x != null);
t1 = x + 5;
check for overflow x + 5
assert(x <= t1 && t1 < x+c);
z = *t1;
\end{lstlisting}

The compiler can recognize that \texttt{x <= t1} is always true,
leading to:-

\begin{lstlisting}
assert(x != null);
t1 = x + 5;
check for overflow of x + 5
assert(t1 < x + c);
z = *t1;
\end{lstlisting}

\section{Incompleteness of static checking}

The invariant checking algorithm is not complete, in that there may be
invariants about the program that are true, but which the invariant
checker does not report as true. First, the invariant checker only uses
declared invariants. An invariant may be true after a particular
statement, but if it is not declared, the invariant checker may not be
able to use it for subsequent statements. If a programmer declares an
invariant x \textless{} y but neglects to declare the invariant y
\textless{} z, the checker will not be able to reason several statements
later that x \textless{} z, even though it may be true about the
program. This is a consequence of the compiler being a checker, not a
theorem prover. Second, the underlying logic used by the checker is not
complete: the invariants require first-order Presburger arithmetic,
complete with disjunction (or) because of the presence of min and max.
The logic used by the checker does not include axioms for disjunction or
min/max.

\section{Partial correctness}

We are pursuing a strategy of being able to prove partial correctness
when these undefined behaviors are possible. Specifically, we would
assume that some undefined behaviors do not occur in some parts of code.
We would be able to prove given this assumption that other undefined
behaviors do not occur in other parts of the program. For example, we
might assume memory allocation and type casts are in fact correct. We
might also assume that unchecked code never reads or writes through
out-of-bounds pointers. We would then be able to provide that checked code
is guaranteed to never read or write through out-of-bounds pointers.

To arrive at more complete correctness guarantees about systems, we will
use the approaches of narrowing the amount of code about which
assumptions have to be made and narrowing the types of assumptions that
have to be made (that type casts are correct and memory allocation is
correct). For example, we will narrow the amount of unchecked code so that
assumptions are needed about less and less code.

\section{Discussion of facts involving members}

\var{fact: }

\begin{quote}
\var{bounds}

\var{var-member-path relop range-exp}

\var{range-exp relop var-member-path}
\end{quote}

\section{Loop invariants and dataflow-sensitive bounds declarations}

\texttt{Array\_ptr} variables with dataflow-sensitive bounds declarations also
need loop invariants if the variables are updated in a loop. Those
invariants must be declared before the loop and cannot occur cannot
occur only within the loop, unless the loop is completely unreachable
from the beginning of the function, in which case it is dead code.

To understand why, suppose the bounds declaration only occurs in the
loop body at or after the read of a variable. There will be a path from
the beginning of the function to the read of the variable. From
condition 2 of consistency, that path must also have a bounds
declaration, which is a contradiction.