\chapter{Bounds Propagation}
\label{sec:propagation}

We have covered bounds expressions on declarations
in~\autoref{sec:checkedc}. The other side of this coin is that we have
to perform propagation to calculate the bounds on pointers at access
time. As pointer bounds expressions are currently flow-insensitive,
this can be done with a set of context-free rules that I will sketch
here, and show fully in~\autoref{app:propagation-rules}.

Our bounds propagation algorithm works in two steps. The first step
normalizes existing bounds expressions into their most general form,
and then the main propagation rules work over these canonical forms to
compute the bounds we should check any pointer access with.

\section{Canonical Forms}
\label{sec:canonical-forms}

We have been careful to define our bounds expressions such that they
can be normalized into a single form, as shown in
\autoref{tab:canonical}. This normalization makes bounds propagation
easier, rather than having to ensure propagation works on every kind
of bounds expression. This should also allow us to add new kinds of
bounds expressions without having to change the propagation algorithm.

Checked array pointers with an explicit bounds expression are not the
only kind of expressions we can compute bounds for. With this
algorithm we may also need to compute the bounds of checked singleton
pointers, and of the pointers derived from checked arrays (exactly
when an array turns into a pointer we will cover shortly).

\begin{table}[ht]
\centering
\begin{tabular}{lll}
\toprule
Bounds Kind & Declaration & Canonical Bounds \\
\midrule
Range & \boundsdecl{\ArrayptrT~p}{\bounds{\mv{l}}{\mv{u}}} &
\bounds{\mv{l}}{\mv{u}} \\
Count & \boundsdecl{\ArrayptrT~p}{\boundscount{\mv{n}}} &
\bounds{p}{p + \mv{n}} \\
Byte Count & \boundsdecl{\ArrayptrT~p}{\boundsbytecount{\mv{n}}} &
\bounds{p}{((\Arrayptr{\kw{char}})p) + \mv{n}} \\
Singleton & \expr{\PtrT~p} & \bounds{p}{p + 1} \\
\addlinespace
\emph{Array} & \expr{\mv{T}~a~\kwchecked[\mv{N}]} &
\bounds{a}{a + \mv{N}} \\
\bottomrule
\end{tabular}
\caption{Canonical Bounds Expressions. In Canonical Bounds, the
\expr{+} refers to C's pointer-integer addition operator, which adds
integer multiples of the size of the pointer's referent type to the
original pointer (hence the cast to \Arrayptr{\kw{char}} in the
\kwbytecount{} case). The bounds on \emph{Array} are used when
\expr{a} is converted from an array into a pointer, and \mv{N} must be
constant.}
\label{tab:canonical}
%%% Local Variables:
%%% mode: latex
%%% TeX-master: "../tr02"
%%% End:

\end{table}

\section{C's Value Semantics}
\label{sec:cs-value-semantics}

Before we can accurately cover our propagation rules, we need to cover
a discussion of how C's lvalues and values work\endnote{What the C11
Specification calls ``values'' are more commonly called ``r-values''
in other language descriptions.}.

In C, there are two kinds of values. Lvalues describe the locations of
objects in memory, and are used to both read from and write to memory.
Integers, floats, and pointers are values.

Every expression in C evaluates to either an lvalue or a value. We
will call an expression that evaluates to an lvalue an \emph{lvalue
expression}; and likewise an expression that evaluates to a value a
\emph{value expression}. Usually expressions in C require their
sub-expressions to be value expressions. Some expressions require
certain of their sub-expressions to be lvalue expressions.
\autoref{fig:lvalue-expressions} shows all lvalue expression types and
all places they may be used in C --- all other expressions are value
expressions and have only value sub-expressions.

As our invariants are all about how programs access memory, and
programs can only access memory using an lvalue expression, our bounds
propagation rules are inextricably tied to the notion of lvalue
expressions and value expressions.

\begin{figure}[ht]
\begin{align*}
  \text{LValue Expressions} \qquad
  e_l \Coloneqq {} & x \tag*{Variables} \\
  \mid {} & \deref e_v \tag*{Dereference} \\
  \mid {} & e_v[e_v] \tag*{Index} \\
  \mid {} & e_l \member m \tag*{Struct Member Access} \\
  \mid {} & e_v \arrow m \tag*{Pointer Member Access} \\[1em]
  \text{Value Expressions} \qquad
  e_v \Coloneqq {} & \addrof e_l \tag*{Address-of} \\
  \mid {} & e_l \assign e_v \tag*{Assignment} \\
  \mid {} & e_l \compassign e_v & \tag*{Compound Assignment} \\
  \mid {} & e_l \inc \mathrel{\mid} \inc e_l \tag*{Increment} \\
  \mid {} & e_l \dec \mathrel{\mid} \dec e_l \tag*{Decrement} \\
  \mid {} & e_v \member m \tag*{Struct Member Access} \\
  \mid {} & \hdots \tag*{Other Value Expressions} \\
  \mid {} & e_l \tag*{Lvalue \& Array Conversion}
\end{align*}%
\caption[Lvalue expressions and their usage in C]{Lvalue expressions
and their usage in C. A more complete description of C's expression
syntax, including a definition of $\binop$ (binary operators), is
contained in~\autoref{app:c-syntax}. Note that a Struct Member Access
can be an lvalue or a value depending on whether the sub-expression is
an lvalue or a value (respectively).}
\label{fig:lvalue-expressions}

%%% Local Variables:
%%% mode: latex
%%% TeX-master: "../tr02"
%%% End:

\end{figure}

The subtle nuance in lvalue expression evaluation is that lvalues may
also appear in a value position in an expression. In this case, one of
two things happens: if the lvalue has array type, the lvalue
expression evaluates to a pointer to the start of that array. If the
lvalue does not have array type, then the lvalue is read to find the
underlying value stored at its location. These operations are
respectively called an \emph{array conversion}, and an \emph{lvalue
conversion}.

Thus C differentiates implicitly between lvalues, which denote
locations, and the objects in those locations. The result of
performing this lvalue conversion is the value in the location denoted
by that lvalue, which we refer to as an \emph{lvalue target}.

\section{Propagation Rules}
\label{sec:propagation-rules}

Our bounds propagation rules, at a broad level, will work with these
three different evaluation models, separately (but mutually
recursively) propagating bounds for value expressions, \emph{value
bounds}, for lvalue expressions, \emph{lvalue bounds}, and for the
value result of an lvalue conversion, \emph{lvalue target bounds}. The
full rules are shown in~\autoref{app:propagation-rules}.

In general, the value bounds of an expression are the value bounds of
the pointer sub-expression with bounds; the lvalue bounds of an
expression correspond to the bounds on the storage location of that
lvalue; and the lvalue target bounds of an expression correspond to
the bounds declared for that lvalue.

During propagation we add two more kinds of bounds annotation:
$\mbounds(\text{any})$, which is equivalent to an infinite range, and
represents the bounds on \NULL{}; and $\mbounds(\text{unknown})$,
which is equivalent to an empty range, and represents that we cannot
calculate a static description of the run-time bounds.

\subsection{Lvalue Bounds}
\label{sec:lvalue-prop-rules}

The propagation rules for lvalue bounds of an lvalue expression are
shown fully in~\autoref{fig:lval-bounds-prop}. The intuition behind
the bounds these rules calculate is that they are the bounds on the
memory used for the lvalue itself, for instance on the stack. The
bounds declarations are not used in lvalue bounds propagation.

\paragraph{Variables} A (non-array-typed) variable only
needs enough space to store one value of the type it holds, so
pointer-typed variables only uses enough space for one pointer. This
is what $\sbounds(e_v)$ computes.

In the case of array-typed variables, the lvalue itself is an array,
so the amount of space used can be deduced from the array size and
type, if the array is constant-sized. This is what $\abounds(e_v, N)$
computes. If the array is not constant-sized, then we cannot deduce the
bounds statically.

\paragraph{Dereferencing and Indexing} In the case of pointer
dereferencing or indexing, the location being accessed is the memory
that the pointer references, which is exactly what the value bounds on
that pointer describe. In the case of an indexing expression,
$e_v[e_v]$, C makes no requirements on which sub-expression is the
pointer, so we define the notion of the $base$ of an expression, which
in this case refers to the pointer-typed sub-expression.

\paragraph{Member Accesses} In the case of struct and pointer memory
accesses, the lvalue bounds we propagate match the location of the
struct member in memory, including if it is an array or not. A further
description of this behaviour is included in~\autoref{sec:narrowing}.

\subsection{Value Bounds}
\label{sec:value-prop-rules}

The propagation rules for value bounds of expressions are shown
in~\autoref{fig:val-bounds-prop}. The intuition behind the bounds
these rules calculate is that these bounds expressions effectively
propagate the lvalue target bounds up from any lvalue conversions.

\paragraph{Null Pointers} The null pointer, \NULL{}, always has
$\mbounds(\text{any})$, as we allow any pointer with bounds also to be
\NULL{}. We will check the pointer is non-null before dereferencing.

\paragraph{Literals} All other literals have unknown bounds, as they
are not pointer literals.

\paragraph{Address-of Operator} The address-of operator converts an
lvalue sub-expression into a pointer representing its memory location.
Thus the bounds on this pointer must be the lvalue bounds of the
lvalue.

\paragraph{Assignment} In C, assignment and compound assignment are
expressions, not statements, so can appear within other expressions.
The resultant value is the same as the value that is assigned into the
lvalue, so in the case of assignment the bounds are the bounds of the
right hand side, and in the case of compound assignment the bounds are
the lvalue target bounds of the left hand side. Increment and
decrement are just special cases of compound assignment.

\paragraph{Struct Member} In the case of struct member expressions, we
narrow to the field of the struct as we do when the base expression is
an lvalue expression. Our implementation does not yet cope with this,
as we have no way of referring to the base value expression from which
we want to calculate the bounds.

\paragraph{Function Calls} For function declarations, we allow the
programmer to specify parameter and return bounds in terms of other
parameters. Therefore, to calculate the bounds on a value returned
from a function call, we need to substitute any argument values into
the bounds expression of the return value.

\paragraph{Comma Expressions} As comma expressions evaluate to their
final expression, we propagate the value bounds of the final
expression.

\paragraph{Array and Lvalue Conversions} In the case of
non-array-typed lvalues, because we will be reading the lvalue in
order for it to become a value, the bounds on this conversion
expression refer to the bounds on the object the lvalue denotes, i.e.
the lvalue target bounds of that lvalue expression. In the case of
array-typed lvalues, evaluation will perform an array conversion,
converting the array into a pointer to its first element, which
corresponds to the lvalue bounds on that array.

\paragraph{Binary Operators} Binary operator expressions returning a
pointer type, such as performing pointer arithmetic, take the bounds
of the base expression, i.e. the sub-expression that has pointer type.
This ensures that for a pointer, $p$, $p + 3$ and $p - 2$ have the
same bounds as $p$ does. The notion of ``base'' here is the same as
with array indexing.

\paragraph{Relational and Logical Operators} Relational operators
return a boolean result, i.e. an integer 0 or 1 corresponding to the
comparison result. These do not denote anywhere in memory, so do not
have bounds. The same applies to the unary negation operator.

\paragraph{Other Expressions} For other expressions, including
arithmetic over integers (but not floats), we propagate bounds from
sub-expressions to expressions. This allows us not to lose bounds
information when performing bit-wise operations over pointer values
(for pointer alignment, for example). The requirement we make is that
if more than one sub-expression has bounds, we can only merge bounds
if they are syntactically equal. We may later relax this to being
syntactically equal modulo variable equality. In the case of float
expressions, we do not believe performing pointer arithmetic with
floating-point numbers makes any sense, so float expressions may not
carry or propagate bounds information.

\subsection{Lvalue Target Bounds}
\label{sec:lvalue-target-prop-rules}

The propagation rules for value bounds of expressions are shown
in~\autoref{fig:lval-target-bounds-prop}. The intuition behind the
bounds these rules calculate is that these bounds represent the extent
in memory of the value returned when an lvalue conversion is
performed.

\paragraph{Checked Singleton Pointers} The bounds of a \PtrT, \mv{p},
start at their value, and include exactly enough space for a single
value of their own type, \mv{T}. They are
$\left[\mv{p}, \mv{p} + \sizeof{\mv{T}} \right)$.

\paragraph{Variables} The lvalue target bounds for a variable declared
with bounds is exactly the bounds expression it was declared with.

\paragraph{Member Accesses} Struct and pointer member accesses both
use the declared bounds on the member to compute their bounds
expressions. In the case of struct member accesses, they use the value
in the same struct of any other members referenced in the bounds
expression. In the case of pointer member accesses, any members
referenced in the bounds expression are also computed using a pointer
member access of the same pointer.

For instance, given a struct, \mv{s}, with two members, \mv{m1} and
\mv{m2}, where \mv{m2} has declared bounds of \boundscount{\mv{m1}},
the lvalue target bounds of $\mv{s} \member \mv{m2}$ are
$\left[\mv{s}\member\mv{m2}, \mv{s}\member\mv{m2} +
\mv{s}\member\mv{m1} \right)$. If we had a pointer, \mv{p}, to the
same type of struct, the lvalue target bounds of
$\mv{p} \arrow \mv{m2}$ would be
$\left[\mv{p}\arrow\mv{m2}, \mv{p}\arrow\mv{m2} + \mv{p}\arrow\mv{m1}
\right)$. In the same way, any variables referenced in expressions on
struct members are assumed to refer to other members of the same
struct.

\paragraph{Dereferencing and Indexing} We cannot propagate lvalue
target bounds for a pointer that is stored at another pointer (unlike
multidimensional arrays). This is because we have no way of expressing
where the bounds of the inner pointer are stored or should be
computed, unlike in a struct where there is potentially storage space.

\section{Narrowing}
\label{sec:narrowing}

One important part of our approach is how we choose to ``narrow''
bounds when accessing members of structs or elements of arrays. This
means potentially inferring a sub-range of the aggregate object's
bounds for accesses to inner parts of that object.

It is this narrowing that ensures the second part of the
well-formedness condition in~\autoref{sec:soundness}, that all
in-lifetime non-null pointers of type \mv{T} with bounds must point to
an object in memory of type \mv{T}.

In the case of structs (and unions), we have to narrow to the bounds
sub-range for a particular member, because different members have
different types. This is what the functions like $\srbounds(e_v, m)$
do, narrowing the bounds inferred to the bounds of the field $m$
within $e_v$.

In the case of arrays, our semantics allow us two implementations. The
first is that we do narrow through each dimension of a
multidimensional array, to ensure an access, \lstinline|a[i][j]|, to a
5 by 5 array, is within the 5 elements starting at
\lstinline|a[i][0]|. Our compiler actually chooses the other
alternative, ensuring that any access to \lstinline|a[x][y]| is within
the 25 elements starting at \lstinline|a[0][0]|. This maintains
well-formedness as arrays all contain elements of the same type, and
should allow the compiler to remove some bounds checks from tight
inner loops.

\section{Propagation Limitations}
\label{sec:prop-limitations}

We have several limitations with the current propagation algorithm,
the most major of which relates to ``modifying expressions'', as
defined in~\cite[Section 3.4]{CheckedCv06}, which pose a problem to
the flow-insensitive algorithm. In particular, our algorithm does not
introduce temporaries for expressions which are modifying, instead
duplicating these expressions into the bounds expressions, and then
computing based off the duplicates. This means if we generate directly
from the bounds expressions, each modifying expression could be
generated in code several times.

For instance, say I have an array, $a$ of structs, each with a member
$m1$ that has bounds defined to be \boundscount{\mv{m2}}, where $m2$
is another member of the same struct. In this case, an access like
\lstinline|a[i++].m1[0]| will have the bounds inferred of
\bounds{a[i++].m1}{a[i++].m1 + a[i++].m2}. If we generate dynamic
checks directly from bounds expressions, we have now incremented
\lstinline|i| four times (once for the pointer itself, once for the
lower bound, twice for the upper bound), which changes the meaning of
the program.

This also is a problem if the programmer calls a function that returns
a struct, and directly (without assigning into a temporary) accesses a
member of the returned struct. We have no way of referring to the
place in memory of the returned struct.

In both of these cases, the programmer can manually insert a temporary
variable to avoid these limitations.

We are planning to solve this issue with a set of dynamically-scoped
keywords that will refer to the current expression value, the current
struct base pointer, and the current return value, which we would then
treat as temporaries during code generation. While it makes the code
generation more complex, it means we can infer the bounds in more
places.

We also totally lack a way of handling ternary expressions during
bounds propagation. Arguably we can generate new bounds expressions
with ternary expressions in both the upper and the lower expressions,
but this will make any static analysis much harder in the long term.
We may introduce a new form of bounds expression to handle these cases
explicitly.



%%% Local Variables:
%%% mode: latex
%%% TeX-master: "tr02"
%%% End:
