% !Tex root = checkedc.tex

\chapter{Bounds declarations for structure types}
\label{chapter:structure-bounds}

This chapter extends reasoning about bounds to objects with structure
types by introducing bounds declarations for structure members.
Structure types may have members with \arrayptr\ types. Those
members must have \emph{member bounds declarations} associated with them in order for
the values stored there to be used to access memory.

The declarations of
structure members may include \keyword{where} clauses that declare member
bounds. The member bounds declarations may also be placed inline.
A structure declaration may also include \keyword{where} clauses at the same
level as member declarations.  Here are examples:

\begin{lstlisting}
struct S {
    array_ptr<int> data where data : count(num);
    int num;
};
\end{lstlisting}
or
\begin{lstlisting}
struct S {
    array_ptr<int> data : count(num);
    int num;
};
\end{lstlisting}
or
\begin{lstlisting}
struct S {
    array_ptr<int> data;
    int num;
    where data : count(num);
};
\end{lstlisting}

Member bounds declarations are program invariants that are assumed to be true
by default for objects of that type.

Making member bounds declarations be invariants provides a way to deal with issues
caused by aliasing. There can be pointers to data structures or members
of data structures. There may be multiple pointers to a single
structure object in memory. When there are multiple pointers,
the pointers are called aliases because they all name the same
memory location.  Aliasing makes it hard to reason about programs.

Consider the example:
\begin{lstlisting}
f(S *q, S *r, bool b)
{
   if (b) {
      q->arr = malloc(sizeof(int)*5);
      q->len = 5;
   }
   else {
      r->arr = malloc(sizeof(int)*5);
      r->len = 5;
   }
}
\end{lstlisting}

Even when b is true, the value \code{r->arr} may still be
changed by a call to f. This can happen when \code{q} and \code{r}
are the same value and are aliases for the same memory location.
Changing one named value (\code{q->arr}) can have the
effect of changing some other value with a distinct name
(\code{r->arr}). In general, it is difficult to know
whether an assignment through one pointer variable is affecting the
members of other pointer variables.

Member bounds declarations being program invariants for structure members allows
localized reasoning about the members. A programmer can assume that the
bounds declarations are true for members of objects of that type.

In the rest of this chapter, to simplify the description, assumptions
about address-taken variables similar to those in
Section~\ref{section:bounds-declarations} are made.
It is assumed that none of the variables or members of variables on the
left-hand side of bounds-declarations have their addresses taken. It is
assumed also that the values of variables or members of variables whose
addresses are taken are not used in bounds expressions.

\section{Declaring bounds for structure members}

Member bounds declarations have the form:
\begin{tabbing}
\var{member}\=\var{-bounds-decl:}\\
\> \var{member-path} \code{:} \var{member-bounds-exp} \\
\\
\var{member-path:}\\
\> \var{identifier} \\
\> \var{member-path} \code{.} \var{identifier}
\end{tabbing}

A member path is a sequence of one or more member names, separated by
the `\code{.}' operator. The sequence of members must be a valid sequence of
member accesses for the structure type. The common case of using a
member of the structure type is simply a member path of length 1.

Member bounds expressions are similar to the bounds expressions
described in Section~\ref{section:bounds-declarations}, 
except that members of the structure type are
used in place of variables in the non-modifying expressions. In
addition, pointer indirection and indirect member references are
excluded.

A structure member whose type is \arrayptr\ or a
checked array type may have at most one bounds declared for it. The
typing rules for member bounds declarations are similar to those for
variable bounds declarations. For bounds declarations of the form
\boundsdecl{\var{member-path}}{\boundscount{\var{e1}}}, the
\var{member-path} cannot have the type \arrayptrvoid\ and
the expression \var{e1} must have an integral type. For bounds declarations
of the form \boundsdecl{\var{member-path}}{\bounds{\var{e1}}{\var{e2}}},
the types of \var{e1} and \var{e2} must be pointers to the same type.
Typically \var{member-path} is also  a pointer to that type or an
array of that type.  However, \var{member-path} can be a pointer to
or an array of a different type.

A structure consists of a list of member declarations, each of which
consists of a type specifier followed by one or more structure member
declarators. Structure member declarators are changed to allow 
optional in-line specification of member bounds and \keyword{where}
clauses.

\begin{tabbing}
\var{struct}\=\var{-member-declarator:}\\
\> \var{declarator where-clause\textsubscript{opt}} \\
\> \var{declarator\textsubscript{opt}} \code{:}
   \var{constant-expression} \\
\> \var{declarator} \code{:} \var{member-bounds-exp}
\var{where-clause\textsubscript{opt}} \\
\\
The list of member declarations is extended to include \keyword{where}
clauses:\\
\\
\var{struct-member-declaration:}\\
\> \ldots{} \\
\> \var{where-clause}\textsubscript{opt}  \\
\\
The remaining syntax for specifying a structure remains unchanged: \\
\\
\var{struct-or-union-specifier:}\\
\> \var{struct-or-union identifier\textsubscript{opt}} \lstinline|{|
\var{struct-member-declaration-list} \lstinline|}| \\
\\
\var{struct-member-declaration-list:} \\
\> \var{struct-member-declaration} \\
\> \var{struct-member-declaration-list struct-member-declaration} \\
\\
\var{struct-member-declaration:} \\
\> \var{specifier-qualifier-list struct-member-declarator-list} \code{;} \\
\\
\var{struct-member-declarator-list:} \\
\> \var{struct-member-declarator} \\
\> \var{struct-declarator-list} \code{,} \var{struct-member-declarator} 
\end{tabbing}

A member bounds expression can use members and child members of the
structure being declared. Any member paths occurring in the member
bounds expressions must start with members of the structure type being
declared. Here is an example of the use of child members:

\begin{lstlisting}
struct A {
   array_ptr<int> data;
};

struct N {
    int num;
};

struct S {
   A a
   where count(a.data) == n.num;
   N n;
};
\end{lstlisting}

Allowing member bounds to use nested members of members complicates
explaining concepts. Sometimes concepts will be explained using member
bounds that use only immediate members and then generalized to handle
nested members.

\section{Bounds-safe interfaces}
\label{section:structure-bounds-safe-interfaces}

Just as existing functions can have bounds-safe interfaces declared for
them, existing structure types can have bounds-safe interfaces declared
for them. This allows checked code to use those data structures and for
the uses to be checked. Existing unchecked code is unchanged.
To create a bound-safe interface for a structure type, a programmer
declares member bounds or interface types for structure members with
unchecked pointer types.

Here is a member bounds declaration for a structure that is a counted buffer
of characters:

\begin{lstlisting}
struct CountedBuffer {
     char *arr : count(len);
     int len;
}
\end{lstlisting}

Here are bounds-safe interface types for members of a structure for binary
tree nodes. The structure contains pointers to two other nodes.  In
checked code, pointer arithmetic on those pointers is not allowed.

\begin{lstlisting}
struct BinaryNode {
    int data;
    BinaryNode *left : itype(ptr<BinaryNode>);
    BinaryNode *right : itype(ptr<BinaryNode>);
}
\end{lstlisting}

If bounds information is declared for one member of a structure with an
unchecked pointer type, it must be declared for all other members of the
structure with unchecked pointer types.

It is important to understand that the \emph{semantics of unchecked
pointers in unchecked contexts does not change even when bounds are declared 
for the pointers}. The declared bounds are used only by checked code that uses the
data structure, when storing checked pointers into the data structure, and when converting
unchecked pointers read from the data structure to checked pointers.  Code in unchecked
contexts that uses only unchecked pointer types is compiled as though the bounds-safe
interface has been stripped from the source code.

\section{Compatibility of structure types with bounds declarations}

The C Standard defines compatibility of two structure types declared in
separate translation units \cite[Section 6.2.7]{ISO2011}.  This definition
is extended to include member bounds declarations and member bounds-safe
interfaces.  If the structure types are completed in both translation
units, for each pair of corresponding members,
\begin{itemize}
\item If the members have unchecked pointer type,
\begin{itemize}
\item If the members both have bounds-safe interfaces, the bounds-safe
interfaces must either both be bounds expressions or both be interface
types. If both have bounds expressions, the bounds expressions must be
syntactically identical after being placed into a canonical form.
If both have interface types, the interface types must be compatible.
\item Otherwise,  one member must have a bounds-safe interface and the
other member must omit a bounds-safe interface, or both members must omit
bounds-safe interfaces.
\end{itemize}
\item Otherwise, both members must have member bounds declarations or both
members must not have member bounds declarations.  If both members have
member bounds declarations, the bounds expressions must be syntactically
identical after being placed into canonical form.
\end{itemize}
