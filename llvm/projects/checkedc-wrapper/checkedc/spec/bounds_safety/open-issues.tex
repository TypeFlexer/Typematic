% !Tex root = checkedc.tex

\chapter{Pending work and open issues}
\label{chapter:open-issues}

This chapter discusses work that needs to be done and open issues.

\section{Pending work}

This section describes work that should be done soon because it is
needed for completeness or to simplify writing code:
\begin{itemize}
\item For structures, add wording that allows a single statement  
to update a variable member whose member bounds is expected to hold after
the statement.  Also add wording to allow a bundle block to do this.
\end{itemize}

\section{Language and library features to be addressed}

\begin{itemize}
\item
  Variable arguments
\item  We require initializers for variables that have checked pointer types that may
be used to access memory.  We also require initializers for variables that have members
or elements that have checked pointers and that may be used to access memory.
We want to relax this requirement so that fewer program changes are needed when 
porting legacy code to Checked C.  We plan to allow delayed initialization by adding a
{\em definite initialization} analysis
similar to those for C\# or Java for variables or members with checked pointer types.
In programs for those languages, a variable may be declared without an initialization value, as
long is it is definitely initialized before it may be used.  A dataflow analysis prescribed by the
language definition is used to determine when a variable is definitely initialized.
\item
  Pointer casts that produce incorrectly aligned pointers have undefined
  behavior, according to the C11 standard. This hole should be filled in
  for checked pointer types. For checked pointer types, we should specify
  either (1) dereferencing an incorrectly aligned pointer shall cause a
  runtime error or (2) the cast itself shall check any alignment
  requirements. For case 1, note that pointer arithmetic for checked pointer types
  is already
  defined to preserve misalignment.

\end{itemize}

\section{Concrete syntax}
 
The current syntax for describing post-conditions places a \keyword{where}
clause after the function parameter list declaration:

\begin{lstlisting}
f( ...)
where cond1 ...
\end{lstlisting}

This syntax might lead to confusion. We might want to adopt an alternate
syntax that makes this clearer. Some suggestions are the keywords
\keyword{on_return} or \keyword{after}:

\begin{lstlisting}
f(...)
on_return cond1 ...

f(...)
after cond1 ...
\end{lstlisting}

Section~\ref{section:alternate-pointer-type-syntax} describes alternate
syntax proposals for pointer types.

\section{Further out work}
\begin{itemize}
\item Allow conditional bounds expressions.   While conditional
expressions are allowed in the non-modifying expressions in a bounds expression, 
there is not a conditional form of bounds expressions.  This is useful for
specifying that an expression only has bounds if some condition is true.
This could be provided by adding a clause to bounds-exp that uses the
same syntax as C conditional expressions:
\begin{tabbing}
\var{bounds}\=\var{-exp:} \\
\>\var{non-modifying-exp} \code{?} \var{bounds-exp} \code{:} \var{bounds-exp}
\end{tabbing}

We would need to add descriptions of introduction and elimination
forms, as well as enhance the rules for checking the validity of bounds.

\item Allow bounds declarations to declare bounds for expressions.
This would allow the system to easily support tagged null pointers.
The bounds for the untagged pointer would be described. 
A tagged pointer could not be used to access memory.  It would have to be
untagged first  to have valid bounds.

\item Consider whether to allow functions to be
parametric with respect to the state of member bounds. This might be
useful to describe a function that sets one of the members of a
structure type and does not affect the state of other members.

\item We could allow signed integer expressions to be invertible
by trying to prove that the expressions are in range, using the
checking of facts in \keyword{where} clauses.
\end{itemize}

