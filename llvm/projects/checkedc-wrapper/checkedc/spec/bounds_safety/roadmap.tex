\chapter{Planned extension features}
\label{chapter:roadmap}
This chapter describes Checked C features that we plan
to implement within the Checked C compiler.

\section{Pointer arithmetic overflow checking}
\label{section:pointer-arithmetic-errors}

For existing unchecked C pointers, the definition of pointer arithmetic is
described in terms of addresses of elements of an array object. The C
Standard \cite{ISO2011} states, that given a pointer p that points to some element i of
an array object, p + j points to the (i+j)th element of that object.
Pointer arithmetic is defined only for pointers to elements of the array
object or one past the last element of the array object.

We take an alternative approach to defining the meaning of pointer
arithmetic and the error conditions for pointer arithmetic. Pointer
arithmetic is allowed to go out-of-bounds and has a well-defined
semantics. Section~\ref{section:new-pointer-types-semantics}
defines the semantics of the new pointer types
directly in terms of byte addresses, instead of with respect to
addresses of elements of an array. Pointer arithmetic that overflows or
involves a null pointer is defined to produce a runtime error.

The new pointer types allow pointer arithmetic that produces
out-of-bounds values. The C definition leaves pointer arithmetic that
produces out-of-bounds values undefined because it is not clear what the
meaning of should be when the pointers are dereferenced. The new pointer
types prevent out-of-bounds pointers from being dereferenced, and solve
this problem another way. In addition, in practice C implementations
often allow pointer arithmetic to produce out-of-bounds values and C
programs end up relying on this implementation-specific behavior. There
is no reason to cause existing code that computes out-of-bounds pointers
but does not dereference them to break when it is converted to use the
new pointer types.

When pointer arithmetic overflows or involves a null pointer, the
resulting value of the expression cannot be used and program execution
stops. If a system provides for error handling, an error handling
mechanism may be invoked to redirect program execution to a new point of
execution that does not use the value of the expression.

Defining pointer arithmetic this way simplifies reasoning about the new
pointer types. Expected identities such as \code{p + 1 > p} now hold
because, if \code{p + 1} overflows, the value cannot be
used. This allows programmers to narrow the bounds for
\arrayptr\ values by incrementing the lower bound or
decrementing the upper bound, even in situations where the bounds are at
the ends of the address space. Later sections describe places where
allowing an undefined value to be used would complicate reasoning about
programs.

If a compiler cannot prove that a new pointer type value is non-null
before a pointer arithmetic operation, it must insert a check.
Similarly, if a compiler cannot prove that a pointer arithmetic
expression for a new pointer type cannot overflow, it must insert a
check. This may slow a typical program down by a slight amount.

\subsection{Semantics of pointer arithmetic for new pointer types}
\label{section:new-pointer-types-semantics}

This section defines the semantics of pointer arithmetic and explains
when overflow occurs in pointer arithmetic. It is assumed that memory is
addressable at the byte level. The order of bits within a byte is not
specified. The order of bytes within built-in types larger than a byte,
such as integers and floating-point numbers, is also not specified.
Pointers shall be treated as addresses of locations of bytes in memory.
The addresses shall be unsigned integers with a defined range of 0 to
\code{UINTPTR_MAX}. The maximum value of a signed integer that can be
added to an address shall be given by \code{INTPTR_MAX} and the
minimum value of a signed integer that can be added to an address shall
be given by \code{INTPTR_MIN}.

For the new
\arrayptrT\ pointer
types, there are distinct operations for addition and subtraction of
pointers by signed and unsigned integers. The operations behave
similarly, but have different overflow conditions for scaling because
the ranges of signed integers and unsigned integers are different.

\begin{itemize}
\item
  First scaling an integer by \sizeof{\var{T}} is
  defined. To scale an integer \var{i}, \var{i} shall be multiplied by
  \sizeof{\var{T}}, producing an integer \var{j}. If
  \var{i} is a signed integer, the scaled result shall be treated as a
  signed integer. If \var{i} is an unsigned integer, the scaled result
  shall be treated as an unsigned integer. For a signed integer, the
  minimum and maximum range for \var{j} shall be \code{INTPTR_MIN} and
  \code{INTPTR_MAX}. For an unsigned integer, the minimum and maximum
  range for \var{j} shall be 0 and \code{UINTPTR_MAX}. If \var{j} is
  outside its valid range, the operation doing the scaling operation
  shall produce a runtime error.
\item
  \var{p} \code{+} \var{i}, where \var{p} is an
  \arrayptrT\ pointer
  and \var{i} is an integer. The integer \var{i} shall be scaled by
  \sizeof{\var{T}}, producing an integer \var{j}. The
  pointer \var{p} will be interpreted as an unsigned integer. The mathematical
  value \var{p} + \var{j} shall be the result of the operation. If
  \var{p} + \var{j} is out of range for a pointer, the operation shall
  produce a runtime error.
\item
  \var{i} \code{+} \var{p}, where \var{p} is an
  \arrayptrT\ pointer
  and \var{i} is an integer, shall be defined as \var{p} \code{+}
  \var{i}.
\item
  \var{p} \code{-} \var{i}, where \var{p} is an
  \arrayptrT\ pointer
  and \var{i} is an integer. The integer \var{i} shall be scaled by
  \sizeof{\var{T}}, producing an integer \var{j}. The
  pointer \var{p} will be interpreted as an unsigned integer. The
  mathematical value \var{p} - \var{j} shall be the result of the
  operation. If \var{p -} \var{j} is out of range for a pointer, the
  operation shall produce a runtime error.
\item
  \var{p} \code{-} \var{q}, where \var{p} and \var{q} are
  \arrayptrT\
  pointers. The two pointers will be interpreted as unsigned integers
  and the mathematical value \var{p} - \var{q} shall be computed,
  producing an integer \var{j}. If \var{j} is out of range for signed
  integer that can be added to an address, the operation shall produce a
  runtime error. If \var{j} is a multiple of
  \sizeof{\var{T}}, the result shall be
  \var{j}/\sizeof{\var{T}}. If \var{j} is not a
  multiple of \sizeof{\var{T}}, then the value shall
  be determined as follows:

  \begin{itemize}
  \item
    If \var{j} is non-negative,
    \var{j}/\sizeof{\var{T}} shall round toward 0.
  \item
    If \var{j} is negative, it shall be implementation-defined whether
    \var{j}/\sizeof{\var{T}} rounds toward 0 or away
    from 0.
  \end{itemize}
\end{itemize}

An important implication of these definitions is that they put a maximum
limit on the number of elements in an array of type \var{T}. It is
\code{UINTPTR_MAX/}\sizeof{\var{T}}. They also put
maximum limits on the number of elements that can be accessed in an
array by a signed integer or an unsigned integer. That leads to limits
on the size of arrays that can be described by some bounds. A signed
integer that must be non-negative can describe an array of
\code{INTPTR_MAX/}\sizeof{var{T}} elements.

\subsection{Expressing pointer arithmetic as integer arithmetic}
\label{section:pointers-as-integers}

During static checking, pointer arithmetic operations will be converted
to use integer arithmetic. This is necessary in C because at times
programmers do explicit size computations that follow the same rules as
pointer arithmetic.

To support this expansion, integer arithmetic operators are extended
with the operators \plusovf, \minusovf, and \mulovf. The
operators interpret pointers as unsigned integers in some range 0 to
\code{UINTPTR_MAX}. An operator produces a runtime error if the value
of its result cannot be represented by the result type:

\begin{itemize}
\item
  \plusovf\ takes an unsigned integer \var{i} and an
  integer \var{j} and produces an unsigned integer in the range 0 to
  \code{UINTPTR_MAX}. Its result is the mathematical value \var{i} + \var{j}.
\item
  For subtraction, there are two forms:

  \begin{itemize}
  \item
    \minusovf\ takes an unsigned integer \var{i} and an
    integer \var{j} as an argument and computes \var{i} - \var{j}, producing an unsigned
    integer in the range 0 to \code{UINTPTR_MAX}. Its result is the
    mathematical value \var{i} - \var{j}.
  \item
    \lstinline|-|\textsubscript{ovf\_diff } takes two unsigned integers \var{i}
    and \var{j} and computes \var{i} - \var{j}, producing a signed integer of type
    \code{ptrdiff_t}. Its result is the mathematical value \var{i} - \var{j}.
  \end{itemize}
\item
  \mulovf\ takes two integers \var{i} and \var{j} (both either
  signed or unsigned) as arguments. It produces an integer whose type is
  the same as the input argument types. Its result is the mathematical
  value \var{i} * \var{j}.
\end{itemize}

Given an expression \code{e1} with a pointer type and an expression
\code{e2} with an integer type, the expansion of \code{e1 + e2} from
pointer arithmetic to integer arithmetic depends on the type of
\code{e2}. The number of bytes to added must be the same kind of
signed or unsigned integer as \code{e2}.

\begin{itemize}
\item
  If \code{e2} is an unsigned integer, \code{e1 + e2} expands to
  \code{e1} \plusovf\ \sizeof{\var{T}} \mulovf\ \code{e2}.
\item
  If \code{e2} is a signed integer, the expansion of \code{e1 + e2}
  must cast \sizeof{\var{T}} to a signed integer. We introduce a
  signed integer type \code{signed_size_t} that is large enough for
  this cast. \code{e1 + e2} expands to \code{e1} \plusovf\
  \code{((signed_size_t)} \sizeof{\var{T}}\code{)} \mulovf\ \code{e2}. This cast is
  necessary because in C, multiplying a signed integer by an unsigned
  integer implicitly converts the signed integer to be an unsigned
  integer.
\end{itemize}

\subsection{Runtime performance implications}

There will be concerns about the effect of overflow checks on the speed
of pointer arithmetic using the new pointer types. These concerns are an
empirical question to be settled after implementing and using the new
pointer types. It is unclear what the actual cost will be. First, there
will be sometimes additional conditions on expressions used in
pointer arithmetic that prevent overflow from occurring. Second,
compiler optimizations often can remove the checks. Third, programmers
can use lightweight invariants to show statically that checks are not
necessary.

In our experience working on an operating system written in managed code
that required these checks, the slowdown was a few percent, even when
\emph{all} arithmetic instructions were checked for overflow too. The
cost of checks was low because it is often the case that checks are
redundant or can be combined into adjacent machine instructions. For
example, in the code \lstinline|t = *p; p += 1;| the first dereference of
\lstinline+p+ implies that \lstinline+p+ is non-null before the increment.
Otherwise, in a typical C environment, the program would fault at
\lstinline+*p+. In the code \lstinline|if (p < hi) { p += 1; }|
the check \lstinline+p < hi+ implies that the increment cannot
overflow. The checks are also inexpensive on superscalar processors:
they can be placed so that they are statically predicted by
processors to never fail.

\section{Relative alignment of \arrayptr\ values}
\label{section:relative-alignment}

(This section should be placed in Chapter~{\ref{chapter:core-extensions}}
when relative alignment has been implemented).

\arrayptrT\ pointers provide
pointer arithmetic on arrays. The bounds for these pointers usually
describe a range of memory that is exactly the size of some array of \var{T}.
The pointers point to an element of the array. In other words, the lower
bound, the upper bound, and pointer are  relatively aligned to the type
\var{T}. Given a lower bound \var{lb}, an upper bound \var{ub}, and a
pointer \var{p}, there should exist some integer \var{count} such that
\var{ub} = \var{lb} + \var{count}. In addition, there is either some
integer \var{index} such that \var{p} = \var{lb} + \var{index},
where addition is C pointer arithmetic, or \var{p} is null.

The type to which a pointer and its bounds are relatively aligned is
called the relative alignment type. By default, the relative alignment
type for a pointer and its bounds is the referent type. However, the
relative alignment type can be overridden by specifying it explicitly as
part of the bounds.  This is described in
Section~\ref{section:representing-relative-alignment}.
This can be used for bounds for the results of pointers casts and
for  \arrayptrvoid\ pointers. The type
\void\ has no defined size, so the default relative alignment is
undefined for \void.

We considered removing the entire concept of relative alignment from the
design to simplify the design.  We decided against that because it would
increase the cost of bounds checking throughout the program.
Section~\ref{section:design-alternatives:always-unaligned} explains
this choice in more detail.

\subsection{Bounds declarations for results of casts between \plainarrayptr\ types}
\label{section:pointer-cast-results}

(This section should be moved to Chapter~\ref{chapter:core-extensions} when
relative alignment is implemented.).

Typically \arrayptr\ pointers and their bounds are relatively
aligned. Together, they represent a view of an array of \var{T}, where
the pointer has type
\arrayptrT. The
bounds specify a range of memory that is exactly the size of some array
of \var{T} and the pointer points exactly at an element of that array. For
example, suppose short ints are 2 bytes in size and
{\boundsdecl{\code{x}}{\bounds{\code{y}}{\code{z}}}}, where the types of \code{x},
\code{y}, \code{z} are \arrayptrinst{\code{short int}} .

This illustration shows 12 consecutive bytes in memory beginning at
address \code{a}\textsubscript{0}, where \code{y} and \code{z} bound a
3-element array and \code{x} points to the middle of the array. The
memory occuped by the 3-element array is shaded in light blue, and the
element pointed to by x is also cross-hatched. The distances in bytes
between \code{x}, \code{y}, and \code{z} are all multiples of 2,
the size of \code{short int}.
\begin{center}
\relalignpic{4}
\end{center}

This simplifies bounds checking because there is no concern during a
bounds check that a pointer will access memory at the end or beginning
of the array that partially overlaps with the bounds. Suppose, for
example, that \code{x} was not relatively aligned to \code{y} and
\code{z} and points at a\textsubscript{7}. The object pointed to by
\code{x} now straddles the array bounds. This illustration shows this:
\begin{center}
\relalignpic{7}
\end{center}

When \code{x} is not relatively aligned, the bounds check for x
becomes more expensive. With relative alignment, the bounds check is
\code{y <= x} and \code{x < z}. Without relative
alignment, the check for the upper bound needs to compute the highest
address that will be accessed using x. For this example, the highest
address accessed will be
\lstinline|(|\arrayptrchar\lstinline|) x + sizeof(short int) - 1|, so the check becomes
\code{y <= x} and \code{x + 1 < z}.

A pointer cast can create a pointer that is not relatively aligned to
the referent type of the pointer type. This can happen when:

\begin{enumerate}
\item
  A pointer is cast to be a pointer to a larger type. For example, if an
  \arrayptrinst{\code{short int}}\ is cast to be an
  \arrayptrint , the resulting pointer
  may not be relatively aligned to its bounds for \code{int}. In the
  first illustration where \code{x} points to \code{a}\textsubscript{4},
  \arrayptrint\ \code{x} is not relatively
  aligned to \code{y} for \code{int}.
\item
  A pointer is cast to be a pointer to a smaller type, and the size of
  the original referent type is not a multiple of the size of the
  smaller type. For example, a pointer to a 12-byte struct may not be
  relatively aligned to its bounds when it is cast to a pointer to an
  8-byte struct.
\end{enumerate}

The use of struct types to illustrate the second case is intentional.
For most C implementations, the second case never happens for casts
involving scalar types. Scalar types are powers of 2 in size (1, 2, 4,
and 8 bytes). This means that for a cast from a larger scalar type to a
smaller scalar type, the larger scalar type will always be a multiple of
the smaller scalar type.

In general, an
\arrayptrT\ that is
relatively aligned for \var{T} is guaranteed to be relatively aligned
for \arrayptrinst{\textit{S}} only
when \sizeof{\var{S}} is a common factor of
\sizeof{\var{T}}. In other cases, programmers need to
supply additional information using program invariants to show that an
\arrayptr\ pointer is relatively aligned to its bounds. Of
course, programmers who are doing casts to use operations on larger
types instead of operations on smaller types (for example, \code{int}
instead of \code{char}) usually already are doing these checks.

\subsection{Representing relative alignment in bounds declarations}
\label{section:representing-relative-alignment}

Bounds expressions are extended with an optional relative alignment
clause to represent situations where an \arrayptrT\ is not relatively
aligned to its bounds for type \var{T}:

\begin{tabbing}
\var{bounds}\=\var{-exp:}\\
\> \var{\ldots{}}\\
\> \bounds{\var{non-modifying-exp}}{\var{non-modifying-exp}}
         \var{relative-alignment-clause\textsubscript{opt}} \\
\\
\var{relative-alignment-clause:}\\
\> \relalign{\var{type}} \\
\> \relalignval{\var{constant-exp}}
\end{tabbing}

This clause is only added to bounds pairs because (by definition) count
expressions always describe pointers that are relatively aligned to
their bounds.  \code{rel_align} and \code{rel_align_value} are
identifiers.  The optional clause specifies a
relative alignment type \var{T} or the required relative alignment in
bytes.  Given
\boundsdecl{\var{x}}{\boundsrel{\var{e1}}{\var{e2}}{\var{T}}},
\lstinline|((|\arrayptrchar\lstinline|)| \var{x} \lstinline|- (|\arrayptrchar\lstinline|)| \var{e1})
        \lstinline|%| \sizeof{\var{T}}\lstinline|== 0| and
\lstinline|((|\arrayptrchar\lstinline|)| \var{e2} \lstinline|- (|\arrayptrchar\lstinline|)| \var{x})
        \lstinline|% sizeof(|\var{T}\lstinline|) == 0| must be true. If the number of bytes is
specified, \sizeof{\var{T}} is replaced by the
constant expression.  The relative alignment clause \relalign{\var{type}}
is just short-hand for \relalignval{\sizeof{\var{type}}}.

\subsection{Effect on bounds checks}

If the default relative alignment has been overridden and
\boundsdecl{\var{e1}}{\boundsrel{\var{e2}}{\var{e3}}{\var{T}}}, the compiler checks whether
\sizeof{referent-type(\var{e1})} is a common factor of \sizeof{\var{T}}
If it is, it inserts the same
runtime check as before. Otherwise, it inserts a runtime check that
\var{e2} \lstinline|<=| \var{e1} \lstinline|&&|
\var{e1} \lstinline|+| \sizeof{referent-type(\var{e1})} \lstinline|- 1 <| \var{e3}.

\subsection{Examples of uses of bounds declarations that specify relative alignment}

Here are examples of the use of relative alignment clauses in
conjunction with pointer casts. In the first example, there is an
\arrayptr\ to raw data consisting of characters. The pointer is
cast to be an \arrayptrint . However,
the relative alignment remains \code{char}:

\begin{lstlisting}
// cast data to be an array_ptr<int> instead
array_ptr<char> raw_data : bounds(lower, upper) = ...
array_ptr<int> data : bounds(lower, upper) rel_align(char) =
    (array_ptr<int>) raw_data;
\end{lstlisting}

In the second example, the code starts with an
\arrayptrint. The pointer is cast to
be an \arrayptrchar. That
\arrayptr\ is then cast back to be an
\arrayptrint . In the second cast, the
\code{rel_align} clause is omitted because the default relative
alignment for an \arrayptrint\ is
\code{int}.

\begin{lstlisting}
array_ptr<int> data : bounds(lower, upper) = ...
array_ptr<char> byte_data : bounds(lower, upper) rel_align(int) = (array_ptr<char>) data;
array_ptr<int> finish : bounds(lower, upper) = (array_ptr<int>) byte_data;
\end{lstlisting}

The third example illustrates a subtlety when an
\arrayptrT\ has a
relative alignment that is larger than the actual size of \var{T}. The
use of pointer arithmetic may require that relative alignment be
lowered. Suppose that the size of \code{short int} is 2 bytes and the
size of an \code{int} is 4 bytes:

\begin{verbatim}
array_ptr<int> d : bounds(lower, upper) = ...
array_ptr<short int> e : bounds(lower, upper) rel_align(int) = (array_ptr<short int>) d;
array_ptr<short int> f : bounds(lower, upper) rel_align(short int) = e + 1;
\end{verbatim}

While \code{e} can have relative alignment of \code{int}, \code{f}
cannot because pointer arithmetic is done at the granularity of
\code{short int}.

The final example illustrates the use of a dynamic check to allow an
\arrayptrchar\ to be cast to a larger type with a larger relative alignment.
It is a function that does a memory copy and uses an optimized aligned copy if
possible.  For simplicity, it is assumed that the memory pointed to by
\code{dst} and \code{src} does not overlap and that \code{sizeof(int)} is 4:
\begin{lstlisting}
void copy(array_ptr<char> dst : bounds(dst, dst + num),
          array_ptr<char> src : bounds(src, src + num),
          size_t num)
{
  if (num % 4 == 0) {
    array_ptr<int> aligned_dst : bounds(dst, dst + num) rel_align(char) =
        (array_ptr<int>) dst;
    array_ptr<int> aligned_src : bounds(src, src + num) rel_align(char) =
        (array_ptr<int>) src;
    int n = num / 4;
    for (int i = 0; i < n; i++) {
      aligned_dst[i] = aligned_src[i];
    }
 }
 else {
    for (int i = 0; i < num; i++) {
      dst[i] = src[i];
    }
 }
}
\end{lstlisting}

However, \lstinline|num % 4 == 0| implies that \code{aligned_dst} and
\code{align_src} are relatively aligned to their bounds, so the
relative alignment can be changed:

\begin{lstlisting}
void copy(array_ptr<char> dst : bounds(dst, dst + num),
          array_ptr<char> src : bounds(src, src + num),
          size_t num)
{
  if (num % 4 == 0) {
    // num % 4 == 0 implies that aligned_dst and aligned_src are
    // relatively aligned to their bounds.
    array_ptr<int> aligned_dst : bounds(dst, dst + num) rel_align(int) =
        (array_ptr<int>) dst;
    array_ptr<int> aligned_src : bounds(src, src + num) rel_align(int) =
        (array_ptr<int>) src;
    int n = num / 4;
    for (int i = 0; i<n; i++) {
      aligned_dst[i] = aligned_src[i];
    }
 }
 else
   ...
}
\end{lstlisting}

Of course, the \code{rel_align(int)} is redundant and can be omitted.

\subsection{Relative alignment and constant counts}

When an \arrayptr\ with a constant count is cast to another
\arrayptr\ type, all the facts about relative alignment are
easily checkable at compile time. This means that a pointer to data can
easily be cast to be a pointer to a larger type. Suppose in the
following example that the size of integers is 4 bytes. A pointer to 8
bytes of characters can be converted easily a pointer to 2 integers:

\begin{lstlisting}
char a[] = "0123456";
array_ptr<char> p : count(8) = a;
array_ptr<int> r : count(2) = (array_ptr<int>) p;
\end{lstlisting}

\section{Alternative designed consider and not used: Removing relative alignment}
\label{section:design-alternatives:always-unaligned}

(This section should be moved to Chapter~\ref{chapter:design-alternatives} when relative
alignment is implemented).

We considered removing the concept of relative alignment from the design in order
to simplify the design. 
Relative alignment is described in Sections~\ref{section:relative-alignment}
and~\ref{section:pointer-cast-results}.  If relative alignment
is removed, however, compilers have to assume that bounds and pointers are
not relatively aligned.   This would result in more costly bounds checks and
more bounds checks in optimized code.

Checks against upper bounds would  take several more instructions.
Given a pointer \var{p} to type \var{T}, \lstinline|*|\var{p} accesses the memory from
\var{p} to \var{p} \lstinline|+| \sizeof{\var{T}} \lstinline| - 1|. Given an upper bound \var{ub}, the
upper bounds check for \var{p} becomes \var{p} \lstinline|+| \sizeof{\var{T}} \lstinline|- 1 <|
\var{ub}, not \var{p} \lstinline|<| \var{ub}.  The
computation of \var{p} \lstinline|+| \sizeof{\var{T}} \lstinline|- 1| would need an overflow 
check also,  so several extra instructions would be added to an upper bounds check, 
not just an extra addition.

There would also be more bounds checks because it would be harder for compiler optimizers
to eliminate upper bounds checks in loops.  Most programmers would write code that
strides through an array using a comparison that
\var{p} \lstinline|<| \var{ub}. The comparison 
\var{p} \lstinline|<| \var{ub} does not imply
\var{p} \lstinline|+| \sizeof{\var{T}} \lstinline|- 1 <| \var{ub}, so the comparison is
not sufficient for a compiler to optimize away
the upper bounds check. A compiler would have to know that a pointer
and bounds are relatively aligned in order to eliminate the upper bounds
check. It would be hard for a compiler to prove this because it would
require interprocedural or whole-program analysis.

\section{Support for by-reference parameter passing of \protect\arrayptr s}

It is important to support by-reference parameter passing of \arrayptr-typed variables
in C.  It is a common way that multiple values are returned from a function.  By-reference parameter
passing of \arrayptr s will lead to programs where the addresses of local variables or struct
members of type \arrayptr\ are taken.

For example, a function that returns a newly-allocated buffer and its length could be written as:
\begin{lstlisting}
void create(ptr<array_ptr<char>> pbuf where *pbuf: count(*len),
            ptr<int> plen)
{ ...
}
\end{lstlisting}
A caller would take the addresses of local variables to use this function:
\begin{lstlisting}
void f(void)
{
    int len;
    array_ptr<char> buf : count(len) = NULL;
    create(&buf, &len);
}
\end{lstlisting}

We can gain insight into this problem by considering a related
problem.   When the address of a variable that occurs in a bounds declaration is 
taken, it is important to ensure that the invariants for the bounds declaration are not
broken accidentally.  Consider:
\begin{lstlisting}
void bad_code(array_ptr<char> buf : count(len), int len) 
{
    ptr<int> plen = &len;
    *plen = 5;
}
\end{lstlisting}
The problem with allowing the address of just \code{len} to be taken is
that the lvalue for \code{len} actually has additional constraints. This
suggests that pointer variables that hold addresses of address-taken variables should
themselves have additional constraints.  For example:
\begin{lstlisting}
void constrained_code(array_ptr<char> buf : count(len), int len) 
{
    ptr<int> plen = NULL;
    ptr<array_ptr<char>> pbuf = NULL where *pbuf : count(*plen);
    bundle {
       plen = &len;
       pbuf = &buf;
    }
    bundle {
       array_ptr<int> a = malloc((sizeof(char)*5);
       *plen = 5;
       *pbuf = a;
    }
}
\end{lstlisting}
There are some interesting issues that arise from this idea:
\begin{compactitem}
\item The bounds on the lvalues for \code{len} and
\code{buf} must be invariant throughout the execution of the program.
While we can put constraints on pointers that point to those lvalues,
we do not know when the lvalues will be updated.
The implication of this is that only local variables with invariant
bounds declarations (Section~\ref{section:invariant-bounds-declarations})
can have their addresses taken.   Local variables
with flow-sensitive bounds declarations cannot have their addresses taken.
The function could alter the bounds declarations for the local variables
so that they disagree with constraints on the pointers.
\item Once the address of a variable involved in a bounds declaration 
is taken, we cannot add new bounds declarations involving the variable.
The pointer-level bounds declarations would not reflect the new constraint.
\item The constraints on the pointers must fully capture 
all the constraints on the underlying variables.
\item Finally, constraints on the pointers cannot be dropped.
\end{compactitem}

The operations that could be done while updating
address-taken variables or assigning through pointers with 
constraints on them must be limited.
To maintain the bounds declaration, we must ensure that
variables are not modified unexpectedly via aliases. 
Without aliasing information, function calls could not
be allowed.  Updates to address-taken variables could
not be intermingled with reads or writes through pointers
with constraints to avoid the possibility of aliasing.

Taking the addresses of members that are used in member
bounds fits nicely into this approach for supporting
by-reference paramters.  The lvalues involved there do have
invariant bounds through the execution of the program,
which is the same requirement that we have placed on
local variables.