% This macro creates a picture of a short int array in memory with upper and 
% lower bound pointers y and z and a pointer to an  element x of the array.
% The macro takes a parameter that is the byte address of x. 
%
% For example, \relalignpic{4} places the pointer to x at byte 4.

\newcommand{\relalignpic}[1]{
\begin{tikzpicture}
% use macros to name constants
\newcommand{\rectsize}{20}
\newcommand{\xindex}{#1}
% the rectangle for the array object
\filldraw[color=cyan!60] (2*\rectsize pt, 0) rectangle(8*\rectsize pt, \rectsize pt);
% the rectangle for the element of the array pointed to by x
\draw[pattern=north east lines] (\xindex * \rectsize pt, 0)
                                rectangle(\xindex * \rectsize +2 * \rectsize pt, \rectsize pt);
% the grid for memory cells
\draw[step=\rectsize pt] (0,0) grid(12*\rectsize pt, \rectsize pt);
% x, y, and z variables
\draw (2* \rectsize pt + \rectsize/2, \rectsize*2.5 pt) node[name=y]{\lstinline|y|};
\draw (\xindex * \rectsize pt + \rectsize/2, \rectsize*2.5 pt) node[name=x]{\lstinline|x|};
\draw (8* \rectsize pt + \rectsize/2, \rectsize*2.5 pt) node[name=z]{\lstinline|z|};
% draw arrows from variables to memory cells
\draw [->, semithick] (y.south) -- ++(0, \rectsize*-1 pt);
\draw [->, semithick] (x.south) -- ++(0, \rectsize*-1 pt);
\draw [->, semithick] (z.south) -- ++(0, \rectsize*-1 pt);
% label memory cells
\foreach \x in {0,...,11} 
{
    \draw (\x * \rectsize pt, - \rectsize / 2 pt) node[anchor=west]{a\textsubscript{\x}};
}
\end{tikzpicture}
} % \newcommand{\relalignpic}



